%----------------------------------------------------------------------------------------
%	PACKAGES AND DOCUMENT CONFIGURATIONS
%----------------------------------------------------------------------------------------

\documentclass{article}

\usepackage[version=3]{mhchem} % Package for chemical equation typesetting
\usepackage{siunitx} % Provides the \SI{}{} and \si{} command for typesetting SI units
\usepackage{graphicx} % Required for the inclusion of images
\usepackage{natbib} % Required to change bibliography style to APA
\usepackage{amsmath} % Required for some math elements 
\usepackage{datetime}
\usepackage[utf8]{inputenc}
\usepackage[polish]{babel}
\usepackage{polski}


\setlength\parindent{0pt} % Removes all indentation from paragraphs

\renewcommand{\labelenumi}{\alph{enumi}.} % Make numbering in the enumerate environment by letter rather than number (e.g. section 6)

%\usepackage{times} % Uncomment to use the Times New Roman font

%----------------------------------------------------------------------------------------
%	DOCUMENT INFORMATION
%----------------------------------------------------------------------------------------

\title{Programowanie CGI \\ Zadanie 6 \\ Komunikator AJAX \\ Temat: Odpowiedź rysowana przez serwer, żądanie rysunku} % Title

\author{Piotr \textsc{Sroczkowski}} % Author name

\date{\today} % Date for the report

\begin{document}

\maketitle % Insert the title, author and date

\begin{center}
\begin{tabular}{l r}
Prowadzący: & dr inż Marek Żabka % Instructor/supervisor
\end{tabular}
\end{center}

% If you wish to include an abstract, uncomment the lines below
% \begin{abstract}
% Abstract text
% \end{abstract}

%----------------------------------------------------------------------------------------
%	SECTION 1
%----------------------------------------------------------------------------------------

\section{Zrealizowane technologie}
\begin{itemize}
\item zaawansowane zmienne CGI lub odpowiedzi http
\item hasła/logowanie się/kryptografia
\item MySql
\item zapis w bazie
\item odczyt z bazy
\end{itemize}

\section{Krótki opis}
Aplikacja umożliwa czatowanie za równo anonimowe (należy zalogować się bez podania hasła,
jest wtedy tworzony tymczasowy użytkownik, usuwany przy wylogowaniu), jak i z identyfikacją
(należy się zarejestrować, weryfikacja mailem).\\

Został użyty composer oraz biblioteka GD.\\
Informacje potrzebne do uruchomienia aplikacji zawiera plik README.md.

Wymyśliłem własny sposób stosunko efektywnego rozwiązania tego zadania. Obrazki są
zapisywane do plików. Na stronie pojawia się 20 obrazków z ostatnimi wiadomościami.
Część z nich to puste obrazki, dla których pliki nie istnieją (jednak nie z błędem,
gdyż są otrzymane nie bezpośrednio po adresie pliku, ale jako żądanie HTTP).
JavaScript co ułamek sekundy odświeża wszystkie obrazki (zmieniając ich atrybut src,
konkretnie dodając dodatkowy parametr do żądania HTTP) oraz chowa puste obrazki.

\end{document}
